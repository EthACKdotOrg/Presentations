\documentclass[16pt]{beamer}
\usepackage[utf8]{inputenc}
\usepackage{pgfpages}
\usepackage{chngpage}
\usepackage{calc}
\RequirePackage[absolute, overlay]{textpos}

\title{All your databases belong to us}
\setbeameroption{show notes on second screen=right}
\input{../common/front}
\begin{frame}
  \titlepage
\end{frame}

\begin{frame}
{Un sysadmin, c'est...}
\begin{itemize}
\item Le mec barbu qu'on ne voit jamais ou presque
\item Le mec barbu qui râle sur les développeurs
\item Le mec barbu sur lequel les développeurs râlent
\end{itemize}
\end{frame}

\begin{frame}
{Mais pas que ;)}
\begin{itemize}
\item Firewall
\item Updates
\item configuration de services
\item ... etc
\end{itemize}
\end{frame}

\begin{frame}
{Pwnage : surface d'attaque}
\begin{itemize}
\item Service mal configuré
\item Accès utilisateurs trop importants/mal gérés
\item Application en "admin party"
\item Faille de sécurité au niveau de l'application
\item Faille de sécurité au niveau système/OS
\end{itemize}
\end{frame}

\begin{frame}
{Réduction du champ}
\begin{itemize}
\item Application Web
\item Serveur Linux
\item Admin pas réveillé
\item Développeur tout aussi peu réveillé
\end{itemize}
\end{frame}


\begin{frame}
{Compromettre un système d'information}
\begin{itemize}
\item Repérer les applications web installées
\item Repérer les versions
\end{itemize}
\note<1>[item]{README}
\note<1>[item]{Changelog}
\note<1>[item]{Autres fichiers courants (install, etc)}
\end{frame}

\begin{frame}
{Faille détectée : exploitation}
\centering
On est dans la mouise...
\end{frame}

\begin{frame}
{Exemple concret : phpMyAdmin}
\begin{itemize}
\item Très souvent présent
\item Bourré de failles
\item Rarement mis à jour et pris en compte
\end{itemize}
\end{frame}

\begin{frame}
{Dernières "features"}
\begin{itemize}
\item CSRF
\item XSS
\item SQL Injection via nom de BDD
\item Directory transversal (teste l'existence d'un fichier)
\item MitM
\item ......
\end{itemize}
\end{frame}

\begin{frame}
{Autre exemple : Elasticsearch}
\begin{itemize}
\item Service configuré pour écouter par défaut sur toutes les interfaces
\item Aucun mécanisme de sécurité
\end{itemize}
\end{frame}

\begin{frame}
{Conséquences : exploitation possible !}
\begin{itemize}
\item Intégration dans un botnet
\item Infection d'autres hosts du réseau
\item Mapping réseau/infra
\item Base pour attaques ultérieures contre l'infra
\end{itemize}
\end{frame}

\begin{frame}
{Conséquences...}
\begin{itemize}
\item Vols de données
\item Perte de clientèle
\item Pertes de temps
\item Blacklisting d'IPs, coupures réseau
\item La fin du monde des Licornes
\end{itemize}
\end{frame}

\begin{frame}
{Comment éviter ?}
\begin{itemize}
\item Communication (devops-bingo-buzz)
\item Réduire la surface
\item Communication (devops-bingo-buzz-bis)
\item Bonnes pratiques
\end{itemize}
\note<1>[item]{Quels droits, quels ports, quels services}
\note<1>[item]{Enlever les références de version, maintenir à jour, etc}
\note<1>[item]{Cycle des mises à jour}
\note<1>[item]{"Château-fort": mauvais (l'ennemi peut aussi être à l'intérieur); Blacklisting: mauvais (préférer whitelisting)}
\end{frame}

\begin{frame}
{Et les attaques type Déni de service ?}
\begin{itemize}
\item "Scalabilité" de l'application
\item Recherche des bottle-necks
\end{itemize}
\vspace{1cm}
\centering
Tests de charge et communication entre départements !
\note<1>[item]{Note: un Déni de service peut ne pas être dû à une attaque...}
\note<1>[item]{Scalabilité: capacité de l'appli à être distribuée sur plusieurs serveurs}
\note<1>[item]{Stateless, etc, etc}
\end{frame}


\input{../common/foot}
\end{document}