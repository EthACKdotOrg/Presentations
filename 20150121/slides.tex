\documentclass[16pt]{beamer}
\usepackage[utf8]{inputenc}

\title{Le coût du gratuit}

\input{../common/front}

\begin{frame}
  \titlepage
\end{frame}

\begin{frame}
{Plan rapide}
\begin{itemize}
\item Le gratuit
\item L'Opensource
\item Alternatives
\item Exemple comparatif
\item Questions
\end{itemize}
\end{frame}

\begin{frame}
{Viendez c'est gratuit !}
\begin{itemize}
\item "Cloud"
\item Services de synchronisation
\item Partage de fichiers
\item Services Mail
\item Services de messageries
\item …
\end{itemize}
\end{frame}

\begin{frame}
{Tout à un coût}
\begin{itemize}
\item Infrastructure serveur
\item Développeurs
\item Marketting
\item …
\end{itemize}
\end{frame}

\begin{frame}
{Gratuit, coût… ?!}
\centering
\includegraphics[height=4cm,keepaspectratio]{./money-question-mark.jpg}
\end{frame}

\begin{frame}
{Sources}
\begin{itemize}
\item Publicités (intégrées)
\item "Freemium"
\item …
\end{itemize}
\end{frame}

\begin{frame}
{}
\centering
Et, surtout, vos données !
\end{frame}

{
\hypersetup{
        colorlinks=true,
        linkcolor=white,
        urlcolor=white
}
\setbeamertemplate{headline} {
        \begin{beamercolorbox}[wd=\paperwidth,dp=8pt,ht=12pt,leftskip=.29cm,rightskip=.29cm]{linecolor}
        Source: \textbf{\href{https://www.facebook.com/business/}{Facebook}}
        \hfill
        \insertinstitute
        \end{beamercolorbox}%
}
\begin{frame}
\centering
\includegraphics[width=\textwidth,height=\textheight,keepaspectratio]{./facebook.png} 
\end{frame}
}

{
\hypersetup{
        colorlinks=true,
        linkcolor=white,
        urlcolor=white
}
\setbeamertemplate{headline} {
        \begin{beamercolorbox}[wd=\paperwidth,dp=8pt,ht=12pt,leftskip=.29cm,rightskip=.29cm]{linecolor}
        Source: \textbf{\href{https://www.google.fr/adwords/?zd=1}{Google}}
        \hfill
        \insertinstitute
        \end{beamercolorbox}%
}
\begin{frame}
\centering
\includegraphics[width=\textwidth,height=\textheight,keepaspectratio]{./google.png} 
\end{frame}
}

\begin{frame}
{Gratuit, vraiment ?}
\begin{itemize}
\item Publicités insérées dans vos messages
\item Publicités insérées dans les résultats de recherches
\item Publicités insérées sur les sites visités
\item …
\end{itemize}
\centering
Et le tracking constant !
\end{frame}

\begin{frame}
{"Dans votre intérêt"}
\begin{itemize}
\item "Optimisations" des recherches
\item Assistant personnel (google now, etc)
\item …
\end{itemize}
\end{frame}

\begin{frame}
{Mais c'est pratique !}
\begin{itemize}
\item Ces "features" vous confortent dans le don de vos données
\item Au final, votre bénéfice est moindre par rapport à celui de ces sociétés
\end{itemize}
\end{frame}

\begin{frame}
{En résumé…}
\centering
\color{red}Si on ne vous vend rien, c'est VOUS que l'on vend !
\end{frame}

{
\hypersetup{
        colorlinks=true,
        linkcolor=white,
        urlcolor=white
}
\setbeamertemplate{headline} {
        \begin{beamercolorbox}[wd=\paperwidth,dp=8pt,ht=12pt,leftskip=.29cm,rightskip=.29cm]{linecolor}
        Source: \textbf{\href{http://spotfire.tibco.com/blog/?p=23638}{Tibco Spotfire}}
        \hfill
        \insertinstitute
        \end{beamercolorbox}%
}
\begin{frame}
\centering
\includegraphics[width=\textwidth,height=\textheight,keepaspectratio]{./money-cloud.jpg} 
\end{frame}
}

\begin{frame}
{Et l'Opensource dans tout ça ?}
\begin{itemize}
\item Aussi gratuit
\item Aussi des applications/services
\item Parfois de la publicité
\end{itemize}

\vspace{1cm}
\centering
¿! Quelle différence ?!
\end{frame}

\begin{frame}
{Tout est question de contrôle}
\begin{itemize}
\item Code source disponible
\item Communauté
\item Contacts avec les développeurs
\end{itemize}
\end{frame}

\begin{frame}
{L'Opensource à l'aide}
\begin{itemize}
\item Une bonne partie des solutions se trouvent dans l'Opensource
\item Des applications libres, ouvertes, pas forcément gratuites
\end{itemize}
\end{frame}

\begin{frame}  
       \begin{tabular}{cl}  
         \begin{tabular}{c}
           \includegraphics[width=3.5cm,keepaspectratio]{./bullshit-meter.jpg}
           \end{tabular}
           & \begin{tabular}{l}
             \parbox{0.5\linewidth}{%  change the parbox width as appropiate
			\begin{itemize}
				\item Hébergé en Suisse
				\item Chiffrement de grade militaire
				\item Notre logiciel breveté
				\item Notre algorithme révolutionnaire
			\end{itemize}
    }
         \end{tabular}  \\
\end{tabular}
\end{frame}

{
\hypersetup{
        colorlinks=true,
        linkcolor=white,
        urlcolor=white
}
\setbeamertemplate{headline} {
        \begin{beamercolorbox}[wd=\paperwidth,dp=8pt,ht=12pt,leftskip=.29cm,rightskip=.29cm]{linecolor}
        Source: \textbf{\href{http://ghostpgp.com/}{GhostPGP}}
        \hfill
        \insertinstitute
        \end{beamercolorbox}%
}
\begin{frame}
\centering
\includegraphics[width=\textwidth,height=\textheight,keepaspectratio]{./military-grade.jpg} 
\end{frame}
}


\begin{frame}
{Alternatives ouvertes au commercial fermé}
\begin{itemize}
\item Facebook : \href{https://diasporafoundation.org/}{Diaspora*}
\item Twitter : \href{http://twister.net.co/}{Twister}
\item GMail : plus compliqué, dépend des besoins
\item DropBox : \href{https://owncloud.org/}{OwnCloud}, \href{https://www.seafile.com/en/home/}{Seafile}
\end{itemize}
\end{frame}

\begin{frame}
{Les sites à connaître}
\begin{itemize}
\item \href{https://prism-break.org/en/}{Prism Break}
\item \href{http://degooglisons-internet.org/}{Dégooglisons Internet}
\end{itemize}
\end{frame}

\begin{frame}
{Les hackerspaces sont là pour aider}
\centering
\href{https://wiki.hackerspaces.org/}{Trouver un hackerspace}

\vspace{1cm}
À Lausanne : \href{https://fixme.ch/}{Fixme}
\end{frame}

\begin{frame}
{Cryptoparties}
Les cryptoparties permettent à tout le monde de comprendre et d’acquérir les compétences.

\vspace{0.5cm}
\centering
\href{https://www.cryptoparty.in/}{Plus d'informations}

\vspace{1cm}
À Fribourg : \href{https://cryptofribourg.ch/}{CryptoFribourg}
\end{frame}

\begin{frame}
{Exemple comparatif}
\center
Signal vs WhatsApp
\end{frame}

\begin{frame}
{Whatsapp}
\begin{itemize}
\item Gratuit (de nouveau)
\item Code fermé
\item Aucun contrôle communautaire
\item Facebook :)
\item Modèle économique vaseux
\end{itemize}
"the company hopes “to work with businesses and organizations you want to hear from.”" \newline

\href{http://uproxx.com/webculture/now-free-again-heres-how-whatsapp-plans-to-make-money-with-no-ads/}{Source}
\end{frame}

\begin{frame}
{Signal}
\begin{itemize}
\item Gratuit
\item Code ouvert
\item Contrôle communautaire
\item Indépendant
\item Chiffrement vérifiable
\end{itemize}
\end{frame}


\begin{frame}
{Tout n'est donc pas perdu !}
\begin{itemize}
\item En employant vous-même des alternatives
\item En parlant des alternatives autour de vous
\item En partageant vos expériences
\item En rapportant des bugs aux développeurs
\end{itemize}
\end{frame}

\input{../common/foot}
\end{document}